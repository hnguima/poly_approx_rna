
\documentclass{article}

\usepackage[portuguese]{babel}
\usepackage[utf8]{inputenc}
\usepackage[margin=1.5in]{geometry}
\usepackage{amsmath}
\usepackage{amsthm}
\usepackage{amsfonts}
\usepackage{amssymb}
\usepackage[usenames,dvipsnames]{xcolor}
\usepackage{graphicx}
\usepackage[siunitx]{circuitikz}
\usepackage{tikz}
\usepackage[colorinlistoftodos, color=orange!50]{todonotes}
\usepackage{hyperref}
\usepackage[numbers, square]{natbib}
\usepackage{fancybox}
\usepackage{epsfig}
\usepackage{soul}
\usepackage[framemethod=tikz]{mdframed}
\usepackage[shortlabels]{enumitem}
\usepackage[version=4]{mhchem}
\usepackage{multicol}
\usepackage{caption}
\usepackage{forloop}

\newcommand{\blah}{blah blah blah \dots}

\setlength{\marginparwidth}{3.4cm}

\newcounter{points}
\setcounter{points}{100}
\newcounter{spelling}
\newcounter{english}
\newcounter{units}
\newcounter{other}
\newcounter{source}
\newcounter{concept}
\newcounter{missing}
\newcounter{math}
\newcounter{terms}
\newcounter{clarity}
\newcounter{late}

\newcommand{\late}{\todo{late submittal (-5)}
\addtocounter{late}{-5}
\addtocounter{points}{-5}}

\definecolor{pink}{RGB}{255,182,193}
\newcommand{\hlp}[2][pink]{ {\sethlcolor{#1} \hl{#2}} }

\definecolor{myblue}{rgb}{0.668, 0.805, 0.929}
\newcommand{\hlb}[2][myblue]{ {\sethlcolor{#1} \hl{#2}} }

\newcommand{\clarity}[2]{\todo[color=CornflowerBlue!50]{CLARITY of WRITING(#1) #2}\addtocounter{points}{#1}
\addtocounter{clarity}{#1}}

\newcommand{\other}[2]{\todo{OTHER(#1) #2} \addtocounter{points}{#1} \addtocounter{other}{#1}}

\newcommand{\spelling}{\todo[color=CornflowerBlue!50]{SPELLING (-1)} \addtocounter{points}{-1}
\addtocounter{spelling}{-1}}
\newcommand{\units}{\todo{UNITS (-1)} \addtocounter{points}{-1}
\addtocounter{units}{-1}}

\newcommand{\english}{\todo[color=CornflowerBlue!50]{SYNTAX and GRAMMAR (-1)} \addtocounter{points}{-1}
\addtocounter{english}{-1}}

\newcommand{\source}{\todo{SOURCE(S) (-2)} \addtocounter{points}{-2}
\addtocounter{source}{-2}}
\newcommand{\concept}{\todo{CONCEPT (-2)} \addtocounter{points}{-2}
\addtocounter{concept}{-2}}

\newcommand{\missing}[2]{\todo{MISSING CONTENT (#1) #2} \addtocounter{points}{#1}
\addtocounter{missing}{#1}}

\newcommand{\maths}{\todo{MATH (-1)} \addtocounter{points}{-1}
\addtocounter{math}{-1}}
\newcommand{\terms}{\todo[color=CornflowerBlue!50]{SCIENCE TERMS (-1)} \addtocounter{points}{-1}
\addtocounter{terms}{-1}}


\newcommand{\summary}[1]{
\begin{mdframed}[nobreak=true]
\begin{minipage}{\textwidth}
\vspace{0.5cm}
\begin{center}
\Large{Grade Summary} \hrule 
\end{center} \vspace{0.5cm}
General Comments: #1

\vspace{0.5cm}
Possible Points \dotfill 100 \\
Points Lost (Late Submittal) \dotfill \thelate \\
Points Lost (Science Terms) \dotfill \theterms \\
Points Lost (Syntax and Grammar) \dotfill \theenglish \\
Points Lost (Spelling) \dotfill \thespelling \\
Points Lost (Units) \dotfill \theunits \\
Points Lost (Math) \dotfill \themath \\
Points Lost (Sources) \dotfill \thesource \\
Points Lost (Concept) \dotfill \theconcept \\
Points Lost (Missing Content) \dotfill \themissing \\
Points Lost (Clarity of Writing) \dotfill \theclarity \\
Other \dotfill \theother \\[0.5cm]
\begin{center}
\large{\textbf{Grade:} \fbox{\thepoints}}
\end{center}
\end{minipage}
\end{mdframed}}

%#########################################################

%To use symbols for footnotes
\renewcommand*{\thefootnote}{\fnsymbol{footnote}}

\title{
\normalfont \normalsize 
\textsc{UNIVERSIDADE FEDERAL DE MINAS GERAIS \\ 
Engenharia Elétrica, 2023} \\
[10pt] 
\rule{\linewidth}{0.5pt} \\[6pt] 
\huge Aproximação Polinomial \\
\rule{\linewidth}{2pt}  \\[10pt]
}
\author{Henrique Nascimento Guimarães}
\date{\normalsize \today}

\begin{document}

\maketitle
\noindent
Disciplina \dotfill Redes Neurais Artificiais \\
Professor \dotfill Frederico Coelho \\

\section{Introdução}

Este trabalho foi realizado utilizando Python 3 e a biblioteca NumPy. O código está disponível em \url{https://github.com/hnguima/poly_approx_rna/}.

\section{Exercícios}

\begin{enumerate}
  
  \item{
    Abaixo estão plotados os gráficos para a aproximação polinomial com grau de 1 a 8, para 10 amostras: \par

    \newcounter{x}
    \forloop{x}{1}{\value{x} < 9}{
      \includegraphics[width=\linewidth,height=10cm]{figures/fig\arabic{x}_10.png} \par
    }
  }
  \item{
    É possível perceber \emph{overfitting} do polinômio de grau 3 até 8, o que é esperado dado que o polinômio de entrada é de grau 2. \par
    Já em graus menores que 2 (grau 1\dots) é possível notar o efeito de \emph{underfitting}.\par 
  }
  \item{
    Abaixo estão plotados os gráficos para a aproximação polinomial com grau de 1 a 8, para 100 amostras: \par

    \forloop{x}{1}{\value{x} < 9}{
      \includegraphics[width=\linewidth,height=10cm]{figures/fig\arabic{x}_100.png} \par
    }

    É possível notar que com o maior número de amostras o efeito de \emph{overfitting} foi menos presente, podendo ser notado em menor nível apenas nos polinômios de grau mais elevado (7 e 8). \par
    O \emph{underfitting} não foi corrigido e é ainda bem aparente no polinômio de grau 1.
  }
  \item{
    O processo de aproximação polinomial pode se assimilar com uma Rede Neural Artificial, considerando que o processo de aproximação por camadas lembra a operação matemática de um \emph{neurônio} de uma RNA.\par
    Outra semelhança notável é que a aproximação por camadas necessita de \emph{treinamento} (assim como uma RNA) e quanto maior a qualidade dos dados amostrados, melhor a aproximação dos seus coeficientes.\par 
    É importante perceber, no entanto, que uma Rede Neural Artificial pode ser (e geralmente será) mais complexa que uma aproximação polinomial. 

  }
\end{enumerate}

\end{document} 